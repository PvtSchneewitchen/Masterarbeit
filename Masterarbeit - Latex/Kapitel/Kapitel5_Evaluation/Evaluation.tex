TODO TEXT MArketing wert einbetten und nicht einzeln

C.LABEL-VR ist die erste Applikation, die als Prototyp im Bereich der Annotation von Punktwolken mit der Virtual Reality Technologie entwickelt wurde. Aus diesem Grund ist es wichtig die Applikation zu evaluieren, um festzustellen, wie die Applikation verbessert werden kann und ob eine Weiterentwicklung lohnenswert ist. Dafür wurde das VR-System aus Kapitel \ref{sec:SystemComponents} bei verschiedenen Gelegenheiten aufgebaut um damit C.LABEL-VR von Probanden mit verschiedenen Voraussetzungen testen zu lassen. Einerseits wurde dabei darauf geachtet, welche Aussagen die Probanden auf die Funktionalität der Applikation trafen. Andererseits soll die Applikation neben der reinen Funktionalität, auch einen gewissen Marketingwert liefern, um beispielsweise potentielle Kunden auf die Firma CMORE aufmerksam zu machen, weswegen auch dahingehend die Aussagen der Probanden beachtet wurden. Die genaue Vorgehensweise bei diesen Tests ist in Kapitel \ref{sec:Methodik} beschrieben. Die Aussagen der Probanden wurden bei diesen Tests analysiert und im Hinblick auf die Voraussetzungen dieser eingeordnet. Das Ergebnis davon wird im Kapitel \ref{sec:EvaluationErgebnis}.

\section{Methodik}
\label{sec:Methodik}
TODO BILD evtl. Bild einer Demo von der Messe 

Das VR-System mit der Applikation C.LABEL-VR wurde mehrmals intern innerhalb der Firma zum Testen bereitgestellt. Darüber hinaus gab es auch Tests auf einer Messe und direkt bei einem Kunden von CMORE vor Ort. Die Aussagen der Probanden bei diesen Tests wurden anhand von deren Voraussetzungen eingeordnet. Die Voraussetzung eines Probanden ergibt sich zum einen über die Einordnung in einer von drei Gruppen. Die erste Gruppe besteht aus Personen, welche keinerlei Erfahrungen mit Punktwolkendaten haben. Die zweite Gruppe besteht aus Personen, die Erfahrungen im Bezug auf Punktwolkendaten haben, jedoch keine zu Annotation von Daten. Die dritte Gruppe besteht aus Personen die sowohl Erfahrungen mit Punktwolkendaten, als auch mit deren Annotation haben. Zusätzlich wurde bei jedem Tester berücksichtigt, ob dieser bereits Erfahrungen mit Virtual Reality gemacht hat. Zu Beginn jedes Tests wurden die wichtigsten Funktionen der Applikation (Navigation, Annotation, User Interface) kurz demonstriert und die Bedienung dazu erläutert. Anschließend hatten alle Probanden die Möglichkeit die Applikation selbst zu testen und dabei Aussagen zur Applikation zu äußern.\\

Während dieser Tests sollten alle Probanden Aussagen über mehrere Dinge treffen die für die Verbesserung und Weiterentwicklung der Applikation relevant sind. Zu Beginn sollte sich jeder Tester zunächst durch die Punktwolke navigieren. Dabei wurden beide Navigationsarten aus Kapitel \ref{sec:Navigation} benutzt und analysiert, wie intuitiv der Proband die Steuerung der jeweiligen Navigationsart empfand und wie präzise er seine gewünschten ziele ansteuern konnte. Ebenfalls wurde untersucht wie stark Symptome der VR-Krankheit während der Navigation auftraten und ob sich diese zwischen den Navigationsmöglichkeiten unterschieden. Anschließend sollten die Probanden einige Punkte der Wolke mit dem Annotationsmöglichkeiten aus Kapitel \ref{sec:Annotation} klassifizieren. Dabei sollten die Tester erläutern, wie gut sie mit der Steuerung des jeweiligen Modus zurechtkamen und wie effektiv sie diesen empfanden. Ebenfalls sollten Aussagen über das User Interface getroffen werden, speziell über die Interaktion mit den Benutzeroberflächen und über die zwei UI-Fenster des Applikationsmenüs aus Kapitel \ref{sec:AppMenu}.\\

Nach dem Testen sollten die Probanden das Potential der Funktionalität von C.LABEL-VR bewerten. Hierbei sollte eingeschätzt werden in wie fern die Umsetzung einer solchen Applikation, mittels Virtual Reality, einen Mehrwert kreiert, um eine Weiterentwicklung der Applikation als lohnenswert zu erachten. Darüber hinaus sollte der Marketingaspekt eingeschätzt werden, speziell wie interessant die Erfahrung mit der VR-Annotation war und welchen Eindruck diese hinterlassen hat. Die Ergebnisse dieser Tests werden im folgenden Kapitel \ref{sec:EvaluationErgebnis} präsentiert.


\section{Ergebnis}
\label{sec:EvaluationErgebnis}
TODO TEXT

In diesem Kapitel werden die Ergebnisse der Evaluation präsentiert. Hierzu werden für jeden Bereich, auf den beim Testen geachtet wurde, die am meisten relevanten Aussagen zusammengefasst und dabei die Voraussetzungen der Probanden berücksichtigt. Die Bereiche, auf die geachtet wurden, sind Navigation, Symptome der VR-Krankheit, Annotation, User Interface und die Einschätzung des Potentials im Hinblick auf Weiterentwicklung und Marketingwert der Applikation. 

\subsection{Navigation}
Im Bereich der Navigation wurden die beiden Navigationsarten aus Kapitel \ref{sec:Navigation} im Bezug auf intuitive Steuerung und Präzision der Steuerung getestet.

\subsubsection{Freier Flugmodus}
Der freie Flugmodus ist die Navigationsart, welche beim Starten der Applikation als Standardnavigation eingestellt ist. Alle Probanden testeten diesen Modus also als erstes. DIe Steuerung dieser Navigationsart hat sich als sehr intuitiv herausgestellt. Keiner der Probanden hatte Probleme damit sich mit der Steuerung zurechtzufinden. Selbst Personen, die bei der anfänglichen Demonstration nicht anwesend waren, also keine genaue Kenntnis über die Bedienung hatten, konnten nach kurzem Ausprobieren die Navigation bedienen. Hierzu war lediglich die Information notwendig, dass die Steuerung mittels der Controller-Sticks erfolgt.\\

Präzise Bewegungen innerhalb von C.LABEL-VR sind zum Beispiel nötig, um sich in die unmittelbare Nähe von Punkten zu navigieren um diese per Touch-Annotation zu Klassifizieren. Die Präzision des freien Flugmodus ist dabei als gut empfunden worden, jedoch gab es die Beobachtung, dass es bei manchen Nutzern einer gewissen Eingewöhnung bedurfte. Dabei wurde festgestellt, dass Probanden mit VR-Erfahrung, welche stets auch Erfahrungen mit anderen Spiele-Medien hatten, deutlich schneller in der Lage waren sich präzise zu Bewegen, als Persohnen ohne VR-Erfahrung.

\subsubsection{Teleport-Modus}
Der Teleport-Modus wurde, im Vergleich zum freien Flug-Modus, als weniger intuitiv Empfunden. Der Hauptgrund dafür sind die schnellen Positionswechsel, welche vor allem die Nutzer überforderten, die keine Erfahrung mit Punktwolken haben. Durch die unverzögerten Positions- und Orientierungswechsel verlieren diese Nutzer schnell die Orientierung innerhalb der Punktwolke. Verstärkt wird dieser Effekt vor allem durch eine Teleport-Distanz zu Beginn des Tests. Um sich mit dieser Navigationsart zurecht zu finden sollte also zu Beginn eine kleine Teleport-Distanz und ein kleiner Drehwinkel gewählt werden.\\

Die Präzision dieser Navigationsart ist ebenfalls schlechter, als die des freien Flugmodus. In den wenigsten Fällen konnten Nutzer, beispielsweise für eine Touch-Annotation, präzise in die Nähe eines Punktes navigieren. Um präziser navigieren zu können musste die Teleport-Distanz verringert werden, was als störend empfunden wurde. Die verringerte Präzision soll aber durch die Verminderung der VR-Krankheitssymptome ausgeglichen werden. Dies ist auch der Fall. Während der Bewegung mittels Teleport-Modus hatte kein Proband stark auftretende Symptome, wie Übelkeit oder Kopfschmerzen.\\

Die Teleportation mittels Pointer wurde, vor allem von Probanden ohne VR-Erfahrung, als kreativ und nützlich Empfunden. Dies kommt also sowohl dem Marketingaspekt, als auch der Funktionalität der Applikation zu Gute. Manche Nutzer bemängelten jedoch die Abstandsschätzung des Pointers, wenn dieser eine große Länge hat.


\subsubsection{VR-Krankheit}

Vor den Tests war zu erwarten, dass Symptome der VR-Krankheit hauptsächlich während der Bewegung durch die Punktwolke auftreten würden. Dabei wurde zusätzlich erwartet, dass die Symptome beim freien Flugmodus häufiger auftreten würden, als beim Teleport-Modus. Diese Annahmen haben sich durch die Tests bestätigt. Nutzer die Übelkeit empfunden haben, bewegten sich stets im freien Flugmodus. Überraschend hingegen war, dass ,von allen Teilnehmern der Tests, lediglich drei Probanden den Test abbrechen mussten, weil sie zu starke Übelkeit empfunden haben. Bei 40 Probanden ergibt dies einen Prozentsatz von 7.5 Prozent. Wie in Kapitel \ref{sec:VRSickness} erwähnt, beträgt die Häufigkeit des Auftretens von VR-Krankheitssymptomen dreißig bis achtzig Prozent. Die Tatsache, dass die Probanden selbst im freien Flugmodus, keine starken Symptome der VR-Krankheit verspürt haben, ist vermutlich auf die Beschleunigung und die Verzögerung dieses Modus zurückzuführen, welche die meisten Probanden auch als angenehm empfanden. Ein weiterer Aspekt ist, dass alle Nutzer, welche den Test abbrechen mussten, keine VR-Erfahrung hatten. Dies lässt vermuten, dass Leute mit VR-Erfahrung, welche schon an eine virtuelle Umgebung gewöhnt sind, weniger anfällig für diese Symptome sind.

\subsection{Annotation}
Im Bereich der Annotation wurden die drei Annotationsmöglichkeiten aus dem Kapitel \ref{sec:Annotation} auf ihre Effektivität und ihren Eindruck auf den Benutzer getestet.

\subsubsection{Pointer Annotation}
Die Annotation mittels Pointer war bei allen Nutzern die, am schnellsten, erlernte Labeling-Methode. Vor allem Probanden mit VR-Erfahrung haben sich schnell mit diesem Modus zurechtgefunden, da die Interaktion mittels Pointern eine gängige Methode innerhalb von vielen VR-Anwendungen ist. Die Effektivität dieser Methode wurde einige Male bemängelt. Nutzer ohne jegliche Punktwolken- oder Annotations-Erfahrung kritisierten die Größe des Pointers. Dieser hätte veränderbar sein sollen, damit ein größerer Pointer verwendet werden kann, um große Flächen an zusammenliegenden Punkten, schnell annotieren zu können. Nutzer mit Punktwolken- und Labeling-Erfahrung bemängelten die Länge des Pointer, welche ins unendliche geht, wenn kein Objekt getroffen wird. Daher kann es dazu kommen, dass bei einem Labeling-Vorgang ein Punkt in weiter Ferne, unabsichtlich annotiert wird.

\subsubsection{Touch Annotation}
Auf Messen wurde die Applikation meist von Mitarbeitern aus dem Verkauf präsentiert. Diese bevorzugten, bei der Präsentation der VR-Annotierung, die Touch-Annotation als Labeling-Möglichkeit. Der Grund dafür war, dass sie sich bei dieser Annotationsart den beeindruckensten Effekt beim Probanden versprachen. Diese Annahme hat sich durch die Tests bestätigt. Vor allem Leute mit Erfahrung in der Annotation von Punktwolken fanden diesen Modus interessant, da er eine Möglichkeit zur Interaktion mit den Punkten bietet, die auf einem Computermonitor nicht vorstellbar ist. Als effektiv wurde dieser Modus nicht angesehen, da jeder Punkt einzeln berührt werden muss, was viel physische Bewegung des Nutzers erfordert.  

\subsubsection{Cluster Annotation}
Von der Effektivität her, wurde die Cluster Annotation am besten bewertet. Speziell Probanden mit Erfahrung in der Annotation von Punktwolken
waren interessiert, diesen Modus in ihre Annotations-Werkzeuge zu integrieren. Auf Wunsch der Algorithmik-Abteilung von CMORE wurde die Bodenpunkt-Analyse innerhalb der Firma präsentiert um sie für den Einsatz in C.LABEL zu evaluieren. Einzig die Steuerung dieses Modus bedurfte für Neulinge eine Gewöhnungszeit, da die Tastenkombination aus drei Tasten nicht intuitiv für sie war.

\subsubsection{User Interface}
Die Interaktion mit den Benutzeroberflächen wurde von allen Probanden als intuitiv empfunden. Grund dafür war, dass die Steuerung dafür die gleiche ist, wie bei der Pointer Annotation. Da die Nutzer in der Regel erst die Pointer Annotation testeten und erst danach mit einer Benutzeroberfläche interagierten, empfanden alle Probanden die Interaktion als leicht erlernbar. 

TODO TEXT Movement Menu zu unübersichtlich





