C.LABEL-VR ist die erste Applikation, die als Prototyp im Bereich der Annotation von Punktwolken mit der Virtual Reality Technologie entwickelt wurde. Aus diesem Grund ist es wichtig die Applikation zu evaluieren, um festzustellen, wie die Applikation verbessert werden kann und ob eine Weiterentwicklung lohnenswert ist. Dafür wurde das VR-System aus Kapitel \ref{sec:SystemComponents} bei verschiedenen Gelegenheiten aufgebaut um damit C.LABEL-VR von Probanden mit verschiedenen Voraussetzungen testen zu lassen. Einerseits wurde dabei darauf geachtet, welche Aussagen die Probanden auf die Funktionalität der Applikation trafen. Andererseits soll die Applikation neben der reinen Funktionalität, auch einen gewissen Marketingwert liefern, um beispielsweise potentielle Kunden auf die Firma CMORE aufmerksam zu machen, weswegen auch dahingehend die Aussagen der Probanden beachtet wurden. Die genaue Vorgehensweise bei diesen Tests ist in Kapitel \ref{sec:Methodik} beschrieben. Die Aussagen der Probanden wurden bei diesen Tests analysiert und im Hinblick auf die Voraussetzungen dieser eingeordnet. Das Ergebnis davon wird im Kapitel \ref{sec:EvaluationErgebnis}.


\section{Methodik}
\label{sec:Methodik}
TODO BILD evtl. Bild einer Demo von der Messe 

Das VR-System mit der Applikation C.LABEL-VR wurde mehrmals intern innerhalb der Firma zum Testen bereitgestellt. Darüber hinaus gab es auch Tests auf einer Messe und direkt bei einem Kunden von CMORE vor Ort. Die Aussagen der Probanden bei diesen Tests wurden anhand von deren Voraussetzungen eingeordnet. Die Voraussetzung eines Probanden ergibt sich zum einen über die Einordnung in einer von drei Gruppen. Die erste Gruppe besteht aus Personen, welche keinerlei Erfahrungen mit Punktwolkendaten haben. Die zweite Gruppe besteht aus Personen, die Erfahrungen im Bezug auf Punktwolkendaten haben, jedoch keine zu Annotation von Daten. Die dritte Gruppe besteht aus Personen die sowohl Erfahrungen mit Punktwolkendaten, als auch mit deren Annotation haben. Zusätzlich wurde bei jedem Tester berücksichtigt, ob dieser bereits Erfahrungen mit Virtual Reality gemacht hat. Zu Beginn jedes Tests wurden die wichtigsten Funktionen der Applikation (Navigation, Annotation, User Interface) kurz demonstriert und die Bedienung dazu erläutert. Anschließend hatten alle Probanden die Möglichkeit die Applikation selbst zu testen und dabei Aussagen zur Applikation zu äußern.\\

Während dieser Tests sollten alle Probanden Aussagen über mehrere Dinge treffen die für die Verbesserung und Weiterentwicklung der Applikation relevant sind. Zu Beginn sollte sich jeder Tester zunächst durch die Punktwolke navigieren. Dabei wurden beide Navigationsarten aus Kapitel \ref{sec:Navigation} benutzt und analysiert, wie intuitiv der Proband die Steuerung der jeweiligen Navigationsart empfand und wie präzise er seine gewünschten ziele ansteuern konnte. Ebenfalls wurde untersucht wie stark Symptome der VR-Krankheit während der Navigation auftraten und ob sich diese zwischen den Navigationsmöglichkeiten unterschieden. Anschließend sollten die Probanden einige Punkte der Wolke mit dem Annotationsmöglichkeiten aus Kapitel \ref{sec:Annotation} klassifizieren. Dabei sollten die Tester erläutern, wie gut sie mit der Steuerung des jeweiligen Modus zurechtkamen und wie effektiv sie diesen empfanden. Ebenfalls sollten Aussagen über das User Interface getroffen werden, speziell über die Interaktion mit den Benutzeroberflächen und über die zwei UI-Fenster des Applikationsmenüs aus Kapitel \ref{sec:AppMenu}.\\

Nach dem Testen sollten die Probanden das Potential der Funktionalität von C.LABEL-VR bewerten. Hierbei sollte eingeschätzt werden in wie fern die Umsetzung einer solchen Applikation, mittels Virtual Reality, einen Mehrwert kreiert, um eine Weiterentwicklung der Applikation als lohnenswert zu erachten. Darüber hinaus sollte der Marketingaspekt eingeschätzt werden, speziell wie interessant die Erfahrung mit der VR-Annotation war und welchen Eindruck diese hinterlassen hat. Die Ergebnisse dieser Tests werden im folgenden Kapitel \ref{sec:EvaluationErgebnis} präsentiert.


\section{Ergebnis}
\section{sec:EvaluationErgebnis}
TODO TEXT

In diesem Kapitel werden die Ergebnisse der Evaluation präsentiert. Hierzu werden für jeden Bereich, auf den beim Testen geachtet wurde, die am meisten relevanten Aussagen zusammengefasst und dabei die Voraussetzungen der Probanden berücksichtigt. Die Bereiche, auf die geachtet wurden, sind Navigation, Symptome der VR-Krankheit, Annotation, User Interface und die Einschätzung des Potentials im Hinblick auf Weiterentwicklung und Marketingwert der Applikation. 

\subsection{Navigation}
Im Bereich der Navigation wurden die beiden Navigationsarten aus Kapitel \ref{sec:Navigation} im Bezug auf intuitive Steuerung und Präzision der Steuerung getestet.

\subsubsection{Freier Flugmodus}
Der freie Flugmodus ist die Navigationsart, welche beim Starten der Applikation als Standardnavigation eingestellt ist. Alle Probanden testeten diesen Modus also als erstes. DIe Steuerung dieser Navigationsart hat sich als sehr intuitiv herausgestellt. Keiner der Probanden hatte Probleme damit sich mit der Steuerung zurechtzufinden. Selbst Personen, die bei der anfänglichen Demonstration nicht anwesend waren, also keine genaue Kenntnis über die Bedienung hatten, konnten nach kurzem Ausprobieren die Navigation bedienen. Hierzu war lediglich die Information notwendig, dass die Steuerung mittels der Controller-Sticks erfolgt.\\

Präzise Bewegungen innerhalb von C.LABEL-VR sind zum Beispiel nötig, um sich in die unmittelbare Nähe von Punkten zu navigieren um diese per Touch-Annotation zu Klassifizieren. Die Präzision des freien Flugmodus ist dabei als gut empfunden worden, jedoch gab es die Beobachtung, dass es bei manchen Nutzern einer gewissen Eingewöhnung bedurfte. Dabei wurde festgestellt, dass Probanden mit VR-Erfahrung, welche stets auch Erfahrungen mit anderen Spiele-Medien hatten, deutlich schneller in der Lage waren sich präzise zu Bewegen, als Persohnen ohne VR-Erfahrung.

\subsubsection{Freier Flugmodus}
Teleport Für neue nutzer oft verwirren, wenn die anfangs teleoprt distanz zu hoch ist. gerade leute ohne punktwolken erfahrung verlieren schnell die orientierung bei den unverzögerten positionswechseln. Genaue positionsansteurung kann nicht gewährleistet werden, das ist aber bewusst. Einschätzung der Pointer länge schwierig. Keine vr.Sickness also wie gewünscht.



VR-Krankheit

Abbruch nur bei 3 Probanden und nur bei free fly
leute ohne vr erfahrung finden es anfangs komisch aber ok
kein problem bei leuten mit vr erfahrung



Annotationsarten
Pointer annotation schnell erlernbar, oft verglichen mit wand anmalen, pointer erscheinung bedarf verbesserung -> pointer zu klein bzw ungeübte personen haben manchmla probleme beim zielen.

Touch annotation, erfüllt den show zweck, haptisches feedback wird als gut angesehen, entfernung zum punkt wird oft bei vr neulingen falsch eingeschätzt -> zu weit weg punkt wird mit hand nicht erreicht

Cluster labeling
großes potential -> auch für weitere algorithmik, labeler akzeptierten auch die ungenauigkeit da immer per hand noch nachgeholfen werden muss, Steuerung für neulinge oft gewühnungsbedürftig durch die 3 tastenkombination

UI
Interaltion war sehr intuitiv jeder hats gleich verstanden, da man das durch pointer labeling schon kennt
Bewegungsmenü zu überladen
Labeling menü gut

Eindruck (coolness faktor)
vor allem sales leute sehen großes potential für messsen und für die abhebung von der konkurrenz
visualisierung wird bei leuten mit point cloud erfahrung beindruckend empfunden




