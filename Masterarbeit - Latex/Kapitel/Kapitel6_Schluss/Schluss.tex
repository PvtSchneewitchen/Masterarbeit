Abschließend sollen die Ergebnisse dieser Arbeit resümiert und ein Fazit gezogen werden. Dazu wird zu den eingangs gestellten Zielen Bezug genommen und ein Ausblick auf mögliche künftige Weiterentwicklungen gegeben.


\section{Zusammenfassung}
TODO TEXT

Ziel dieser Arbeit war es, den Prozess der Datenannotation von dreidimensionalen Punktwolken zu verbessern. Dazu wurde ein Virtual Reality System zusammengestellt und eine Applikation für diese Plattform entwickelt, mit der es möglich ist, 3D-Punktwolken in einer virtuellen Umgebung zu annotieren. Die Funktionalitäten, welche von der Applikation erwartet wurden, sind von CMORE Automotive, dem betreuenden Unternehmen dieser Arbeit, gestellt worden und lauteten folgendermaßen (ausführlich in Kapitel \ref{sec:Goal}): \\

\begin{itemize}
\item Import eines Datenformates für Punktwolken\\
\item Generierung einer VR-Punktwolke aus den eingelesenen Daten\\
\item Bereitstellung einer Navigationsmöglichkeit\\
\item Bereitstellung einer Annotationsmöglichkeit\\
\item Export der Annotierten Daten in das Ausgangsdatenformat \\
\end{itemize}

%In dieser Arbeit wurde ein Virtual Reality System zusammengestellt und eine Applikation dafür entwickelt, um den Prozess der Datenannotation von dreidimensionalen Punktwolken zu verbessern. 
%
%Um den Prozess der Annotation von 3D-Punktwolken zu verbessern, wurde im Zuge dieser Arbeit ein System zusammengestellt und eine, dafür ausgelegte, Applikation entwickelt um die Annotation von 3D-Punktwolken mittels Virtual Reality zu ermöglichen. Dabei wurde versucht die Ziele aus Kapitel \ref{sec:C.LABEL}, welche an dieses Projekt gestellt waren, zu erreichen.


\section{Weiterentwicklungsmöglichkeiten}
TODO TEXT